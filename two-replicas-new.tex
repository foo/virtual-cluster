\subsection{Two replicas without Multiple Assignment}\label{ap:tworep-ni}

We now show that~$\RS(2)+\FP+\CC+\BW$ is even NP-hard without multiple
assignment.
The proof is similar in spirit to proof of hardness of $\RS(2)+\FP+\MA$.

The reduction (Theorem~\ref{th:ma-reduction}) unfolds in two stages.
First, given a solution $S_{\TDPM}$ to $I_{\TDPM}$, we construct a solution $S_{\VCEMB}$ to $I_{\VCEMB}$.
This part is the easier of the two, and mainly consists of placing nodes in Triple Gadgets for triples chosen in $S_{\TDPM}$.

In the second stage, given $S_{\VCEMB}$, we construct the $S_{\TDPM}$.
Again, we use the technique that we call ``families of chunk types'', which was introduced in previous section.
The main technical difficulty lies in controlling the number of nodes that are spawned in certain parts of (asymmetric) tree.
To guantee the desired number of spawned nodes, we use the bandwidth constraints.
Namely, if the number of nodes to be spawned in a subtree is $k$, we set the bandwidth constraints on the uplink of the subtree to $k\cdot (m - k)$, where $m$ is the total number of machines to spawn in the instance.
As we further see in Lemma~\ref{lem:bandwidth1}, such bandwidth constraint in form of a quadratic expression provides both lower- and upper-bound on the number of machines spawned in such subtree.
To see this, consider a simple example: regardless of the bandwidth constraint on the uplink of the subtree, capacities are not exceeded in at least two scenarios: with all $m$ nodes spawned in the subtree, and with $0$ nodes spawned in the subtree.
More precisely, bandwidth constraints in such form excludes configurations with number of nodes between $k$ and $m-k$.

However, we are interested only in lower-bounding the number of nodes to spawn in a subtree, and in fact the upper-bound on the number of nodes is only a liability.
We make sure that the upper-bound on the number of nodes is always satisfied by artificially increasing the number of total nodes to be spawned.
In this way the upper-bound on number of nodes always exceeds the number of leaves of any subtree in which we would like to have $k$ nodes spawned, see Lemmas~\ref{lem:bandwidth2} and \ref{lem:bandwidth3}.
Additional nodes do not interfere with the rest of the construction, as we provide unique chunks for them to process.



\noindent \textbf{Construction.}

For arbitrary instance~$I_{\TDPM}$ of~$\TDM$ we construct a~$\RS(2)+\FP+\CC+\BW$ instance~$I_{\VCEMB}$ the way described in the remainder of this section.
Let $n = |X|=|Y|=|Z|$.
By $T$ we denote the set of all triples of $I_{\TDPM}$, and let $t = |T|$.
For each $e\in X\cup Y\cup Z$, by $T_e$ we denote the set of all triples that contain element $e$.
Let $\deg(e) = |T_e|$, and note that $\sum_e \deg(e) = 3\cdot t$.

We proceed with the construction as follows.

\emph{Chunk Types.}
We construct two sets of chunk types.
The first set corresponds to elements of the universe (that is, $X\cup Y\cup Z$).
The construction of such chunk types is similar to construction of chunk types in Section~\ref{ssec:fprscc}, but to take into consideration the restricted replication factor, we construct the familiy of chunk types (as described in the introduction to this section).
Namely, for each element of universe $e$, we construct as many chunk types as there are occurences of $e$ in triples of $I_{\TDPM}$.
Each such chunk type has exactly two replicas.

The other set of chunk types has one replica, therefore those are called called \emph{unique} chunks.
For unique chunks we simply co-notate the chunk type with chunk replica.

Formally, the construction of chunk types and replicas unfolds as follows:
\begin{enumerate}
  \item For each triple $\tau \in T$, we construct $3$ chunk types, with two replicas each.
  We construct different chunk types for each triple $\tau$, which contain element $e$ (in total $\deg(e)$ chunk types).
  We refer to those replicas by $ch_1(e, \tau)$ and $ch_2(e, \tau)$.
  In total we construct $2\cdot \sum_e\deg(e) = 6\cdot t$ chunk replicas.
  \item Additionally, we construct
$\max\{3\cdot t + 3\cdot n + 1, \sum_e(2\cdot \deg(e)-1)\}$
chunk types called \emph{unique chunks}. We
refer to the set of unique chunks by~$U$.
\end{enumerate}

\emph{The substrate network.}
\begin{enumerate}
  \item The physical network consists of three subtrees connected to
  the root: the {\MatchSubtree}, the {\CoverSubtree}, and a
  {\UnqSubtree}. In the {\MatchSubtree} we put $t$
  {\TripleGadgets}. The {\CoverSubtree} consist of~$n$ element gadgets.
  \item The {\UnqSubtree} consist of~$|U|$ leaves, and two middle nodes:
  a lower and an upper middle node. Note that this is different from $\RS(2)+\FP+\MA(4)$ NP-completeness proof, where {\UnqSubtree} was placed in the {\MatchSubtree}.
  \item \TripleGadget: For each triple, we create a subtree
  consisisting of four vertices: three leaves and one triple root.  We
  attach the root of the triple to the root of the matching subtree.
  \item \ElGadget: For each element~$e \in X\cup Y\cup Z$, we
  construct a subtree consisting of the root of the element (attached
  to the root of the cover subtree), and~$\deg(e)$ leaves.
\end{enumerate}

\emph{Chunk placement.}
The chunks are placed as follows:
\begin{enumerate}
  \item \emph{Chunks in the Matching Subtree:} In {\TripleGadget} of triple $\tau$ we put
  three replicas:
 ~$ch_1(e_X(\tau), \tau), ch_1(e_Y(\tau), \tau), ch_1(e_Z(\tau), \tau)$, one per each leaf.
  \item \emph{Chunks in the Unique Subtree:} We place replicas
 ~$U$ at the leaves of \UnqGadgets.
 \item \emph{Chunks in Element Gadgets:} Consider the \ElGadget{} for the element $e \in X\cup Y\cup Z$.
 We place two types of replicas in the leaves of the gadget.
 We put replicas $ch_2(\tau, e)$ for each $\tau \in T_e$.
 In total, we place $\deg(e)$ replicas, one per each leaf of the gadget.
\end{enumerate}

\emph{Bandwidth constraints.} We use bandwidth constraints of the form
$\Band(k) := k\cdot(\numNodes - k)$, where $\Vms$ is the total number of nodes to be spawned in the instance. Namely, we set the bandwidth
constraints of an uplink of an {\ElGadget} for each element~$e$ to 
$\Band(\deg(e)-1)$, the bandwidth of an uplink of a~$\MatchSubtree$ to 
$\Band(n)$, and an uplink of a~$\CoverSubtree$ to 
$\Band(\sum_e (\deg(e)-1)$.
Note that out of $\deg(e)$ leaves of \ElGadget{} for element $e$, we allow to spawn $\deg(e)-1$ nodes.

\emph{The threshold value and other properties of the instance.} We set the
cost threshold for any solution to the following value:

\begin{footnotesize}
\begin{align*}
  \Thr  & = 2\cdot (3\cdot \numNodes + \sum_e (\deg(e) - 1)) & \mbox{(over 2 hops)}\\
        & + 4\cdot (n\cdot (3 \cdot (3\cdot \numNodes - 3)) / 2) & \mbox{(over 4 hops in {\MatchSubtree})}\\
        & + 4\cdot (\sum_e((\deg(e) - 1)\cdot (\sum_{f\neq e} \deg(f) - 1)/2)) & \mbox{(over 4 hops in {\CoverSubtree})}\\
        & + 6\cdot (3\cdot \numNodes \cdot \sum_e(\deg(e) - 1)) & \mbox{(between {\MatchSubtree} and {\CoverSubtree})} \\
        & + |U|\cdot (|U|-1)/2 & \mbox{(inside {\UnqSubtree})} \\
        & + |U| \cdot (3\cdot n + \sum_e (\deg(e)-1) & \mbox{({\UnqSubtree} to other nodes)}
\end{align*}
\end{footnotesize}
where $\Vms$ is the total number of nodes to be spawned in the instance.
  We
set~$\CostTrans$, the cost of chunk transportation to $\Thr+1$ (so that no chunk transportation happens in any feasible solution), 
$\CostCom = 1$, and we host only one node per machine. We set the
number of machines to spawn to:
$\numNodes := 3\cdot n + \sum_e (\deg(e)-1) + |U|$.
\\

\noindent \textbf{Properties of the substrate network.}
\begin{lemma}
  Assume we have a~$\RS(2)+\FP+\CC+\BW$ instance~$I$ with a subtree
 ~$T'$ with~$l$ leaves and the bandwidth capacity on uplink of~$T'$ is
 ~$\Band(k)$. Assume that no chunk transportation is allowed
  ($\CostTrans = \infty$, so every node must be collocated with the
  chunk it processes in every feasible solution), and~$\CostCom = 1$.
  Then in any feasible solution the number $s$ of nodes spawned in~$T$ satisfies~$s \leq k \vee \Vms-s\leq k$, and~$s \leq l$.
  \label{lem:bandwidth1}
\end{lemma}

\begin{proof}
It holds that ~$s\leq l$ as we cannot spawn more nodes than leaves.
  The bandwidth allocation on the uplink of~$T'$ is
 ~$uplink(s,T) := s\cdot (\Vms - s)$, as no chunk transportation
  is allowed ($\CostTrans = \infty$), and every node in~$T$ has to
  communicate over~$T'$'s uplink with nodes spawned outside of
 ~$T'$. Therefore, in every feasible solution we have:
 ~$uplink(s, T') \leq \Band(k)$.  Let's define the remaining bandwidth
  on the uplink of $T'$~$\remainBw(s) := \Band(k)-uplink(s, T') = s^2 - s\cdot \Vms -
  k^2+k\cdot \Vms~$.
  Every feasible solution fulfills~$\remainBw(s) \geq 0$, which is true for
 ~$s \leq k \vee \Vms-s\leq k$ (follows from the properties of the
  quadratic function).
\end{proof}


Next, we show how to precisely control the number of nodes in the
constructed subtree.

\begin{obs}
  In every feasible solution we have exactly~$|U|$ nodes spawned in a
  {\UnqSubtree} (no chunk transportation is allowed, and every chunk
  type must be processed).
  \label{obs:unq-full}
\end{obs}


\begin{lemma}
  The following properties holds in $S_{\VCEMB}$:
  \begin{enumerate}
    \item The number of nodes spawned in a {\MatchSubtree} is
   ~$3\cdot n$.
    \item The number of nodes spawned in a {\CoverSubtree} is
   ~$\sum_e(\deg(e)-1)$
  \end{enumerate}

  \label{lem:bandwidth2}
\end{lemma}

\begin{proof}
  From Observation~\ref{obs:unq-full} we know that we have~$|U|$ nodes
  in the {\UnqSubtree}. Let's refer to the number of nodes spawned in
  a {\MatchSubtree} by~$M$, and to the number of nodes spawned in
  {\CoverSubtree} by~$C$. By applying Lemma~\ref{lem:bandwidth1} to
  {\MatchSubtree}, we know that: $ M \leq 3\cdot n \vee M \geq \Vms - 3\cdot n$.
  We observe that~$\Vms - 3\cdot n$ is greater than the number of
  leaves in a {\MatchSubtree}.  By applying Lemma~\ref{lem:bandwidth1}
  to the {\CoverSubtree} we know that:~$ C \leq \sum_e(\deg(e)-1) \vee C$ $\geq \Vms - \sum_e(\deg(e)-1)$.
  We observe that~$\Vms - \sum_e(\deg(e)-1)$ is greater than the number
  of leaves in the {\CoverSubtree}.
  We also know that~$\Vms = |U| + C + M$. Therefore, by the pigeon-hole principle
 ~$C = \sum_e(\deg(e)-1)$ and~$M = 3\cdot n$.
\end{proof}


\begin{lemma}
  In $S_{\VCEMB}$ the number of nodes spawned in Element Gadget of
  element~$e$ is~$\deg(e)-1$.
  \label{lem:bandwidth3}
\end{lemma}

\begin{proof}
  Let's call the number of nodes spawned in the Element Gadget of
  element~$e$ the $x_e$.  From Lemma~\ref{lem:bandwidth1}, we know that
 ~$x_e \leq \deg(e) - 1 \vee x_e \geq \Vms - \deg(e) + 1$. We observe
  that~$\Vms - \deg(e) + 1$ is greater than the number of leaves of the
  gadget, which is~$\deg(e)$.  From Lemma~\ref{lem:bandwidth2}, we know
  that the number of nodes spawned in the entire {\CoverSubtree} is
 ~$\sum_e (\deg(e)-1)$. Therefore, by the pigeon-hole principle, we have
  that~$x_e = \deg(e)-1$.
\end{proof}

From the above lemmas we know the precise number of nodes spawned in
certain parts of the tree. Feasible solutions only differ in 
the choice of the~$\deg(e) - 1$ out of~$\deg(e)$ chunks
in each Element Gadget, and the placement of nodes in the
{\MatchSubtree}.

%
% \begin{lemma}
% 
%   Assume we have a~$\RS(2)+\FP+\CC+\BW$ instance~$I$.  Assume that
%   no chunk transportation is allowed, and~$\CostCom = 1$.
%
%   Assume that we have a subtree~$T$ with subtries
%  ~$S_1, S_2, \ldots S_a$ attached to the root of~$T$. Assume that in
%   every feasible solution we have exactly~$Q$ machines in~$T$ (and
%   the rest~$\numNodes - Q$ machines outside of~$T$). The bandwidth
%   capacity on uplink of~$S_i$ is set to~$\Band(x_i)$, where
%  ~$\sum_i x_i = Q$. Then, in every feasible solution the number of
%   nodes spawned in~$S_i$ is~$x_i$.
%
%   \label{lem:bandwidth2}
% \end{lemma}
% 
%
% \begin{obs}
%   \label{obs:nodes-match-cover}
%   Using lemma \ref{lem:bandwidth2} with~$T$ being the whole tree,
%   and~$Q = \numNodes$ we conclude that
%   \begin{enumerate}
%     \item In every feasible solution there are~$3\cdot \numNodes$
%     nodes spawned in {\MatchSubtree}.
%     \item In every feasible solution there are~$\sum_e (\deg(e)-1)$
%     nodes spawned in {\CoverSubtree}.
%   \end{enumerate}
% \end{obs}
%
% \begin{obs}
%   \label{obs:deg-min-one}
%
%   Using lemma \ref{lem:bandwidth2} with~$T$ being the
%   {\CoverSubtree}, and~$Q = \sum_e (\deg(e)-1)$ we conclude that in
%   every feasible solution the number of nodes spawned in {\ElGadget}
%   of element~$e$ is exactly~$\deg(e)-1$.
%
%   Therefore, at least one of~$\deg(e)$ chunks that correspond to
%   element~$e$ was processed in {\MatchSubtree}.
% \end{obs}
%

Similar in spirit to the NP-completeness proof of~$\RS(2)+\MA(4)+\FP$,
we call the {\TripleGadget} active if it contains exactly three nodes. 
Similarly, we call the {\TripleGadget} inactive if it
does not contain spawned nodes, and \emph{partially active} if it
has one or two
spawned nodes.

\begin{lemma}
  In $S_{\VCEMB}$, we have exactly~$n$ active
  {\TripleGadgets}.
  \label{lem:full-or-empty}
\end{lemma}

\begin{proof}
  Since~$I$ is feasible, we know that it has a solution~$\Sol$ of
  cost~$\leq \Thr$.
  By Lemma~\ref{lem:bandwidth2}, we know that there are
  exactly~$3\cdot n$ spawned nodes in the {\MatchSubtree}. Therefore, by
  the pigeon-hole principle, we know that we have at most~$n$
  active {\TripleGadgets}. It remains to show that there
  are no partially active {\TripleGadgets} in the solution of cost
 ~$\leq \Thr$.
  Using Lemma~\ref{lem:bandwidth3}, 
  we conclude that the communication cost of
  nodes in the {\CoverSubtree} is the same for every feasible solution
  (let's name that cost~$P$). We also know that the communication cost
  between nodes in {\CoverSubtree} and {\MatchSubtree} is the same for
  every feasible solution (let's name it~$Q$). Let's call the
  would-be cost of communication in the {\MatchSubtree}, if there were
  exactly~$n$ active gadgets,~$R$.
  The threshold value was chosen so that~$\Thr = P+Q+R$. If we have at least one partially active
  gadget, then the cost of communication in {\MatchSubtree} is greater
  than~$R$, because we increase the number of 4-hop communications by
  at least one per each partially active gadget in comparison to a solution
  where we have exactly~$n$ active gadgets.
\end{proof}

\noindent \textbf{The reduction.}
Using the properties of the substrate network, we perform the reduction in the following way.

\begin{theorem}
 ~$\RS(2)+\FP+\CC+\BW$ is NP-hard.
\end{theorem}

\begin{proof}
  Let's take any instance $I_{\TDPM}$ of~$\TDPM$.
  We show that~$I_{\VCEMB}$ has a solution of cost~$\leq \Thr$ if and only if~$I_{\TDPM} \in \TDPM$ (there exists a perfect 3D matching in $I_{\TDPM}$).


  Let's take an instance~$I$ of~$\TDPM$ and construct an instance~$I'$
  of~$\RS(2)+\FP+\CC+\BW$ in the way described above.  We show that
 ~$I'$ has solution of cost~$\leq \Thr$ if and only if~$I \in \TDPM$
  (there exists a perfect 3D matching).

  ($\Leftarrow$) Let's take any feasible solution~$S_{\TDPM}$ to~$I_{\TDPM}$ and
  produce a solution~$S_{\VCEMB}$ to~$I_{\VCEMB}$. We show that the cost of~$S_{\VCEMB}$ is
  indeed~$\leq \Thr$.
  For each triple~$t\in T$ in~$S_{\TDPM}$, we put~$3$ nodes at
  leaves of triple gadgets corresponding to those triples.  In each
  element gadget (that corresponds to element~$e$), we put~$\deg(e)-1$
  nodes. In each element gadget there is only one leaf without the
  node placed in it: this node contains the chunk replica that is
  processed in the {\MatchSubtree}.
  It is easy to see that~$S_{\VCEMB}$ has cost exactly~$\Thr$ and no
  bandwidth constraint is violated. Each chunk type is processed.

  ($\Rightarrow$) Let's take any feasible solution~$S_{\VCEMB}$ to~$I_{\VCEMB}$ and
  produce a solution~$S_{\TDPM}$ to~$I_{\TDPM}$ by taking triples that correspond
  to active triple gadgets. Using Lemma~\ref{lem:full-or-empty}, we
  conclude that there are exactly~$n$ active triple gadgets. By
  feasibility of~$S_{\TDPM}$, we know that each chunk type is
  processed. From Lemma~\ref{lem:bandwidth3}, we know that out
  of~$\deg(e)$ chunk types that correspond to~$e\in X\cup Y\cup Z$,
  exactly one is processed in the {\MatchSubtree}, hence each
  element of~$X\cup Y\cup Z$ is matched.
\end{proof}

