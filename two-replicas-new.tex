We have seen that replica selection flexibilities can render embeddings computationally hard.
We will now provide a more detailed look at this hardness result
and explore the minimal requirements for rendering replica selection hard.
In particular, we will show that already two replicas for each chunk type are sufficient to
introduce intractability.

Namely, we provide the NP-hardness results for two restricted variants of Virtual Cluster Embedding (Sections~\ref{ap:tworep-ma} and \ref{ap:tworep-ni}).
We augment the $\RS$ variant of $\VCEMB$ problem in the following way: by $\RS(k)$ we denote the problem where each chunk has the redundancy factor at most $k$.
In Section~\ref{ap:tworep-ma} we provide the hardness result for $\RS(2)+\MA+\FP$, and in Section~\ref{ap:tworep-ni} we provide the hardness result for $\RS(2)+\FP+\CC+\BW$.

Both problems are reduced from the problem $\TDPM$ (see Section~\ref{sec:3dm_intro} with no further restrictions.
The constructions are based upon the reduction of $\TDPM$ to $\FP+\RS+\MA$ (see Section~\ref{ssec:fprsma}) and the reduction of $\TDPM$ to $\FP+\RS+\CC$ (see Section~\ref{ssec:fprscc}).
However, in contrast to Section~\ref{ssec:fprscc}, in two replica variant without multiple assignment, we added the bandwidth constraints.
It is currently unknown to the authors of this very paper, whether the hardness result holds without bandwidth constraints (namely, whether the problem $\RS(2)+\FP+\CC$ is NP-hard).
The necessity for bandwidth constraints arises as to deal with restricted factor of replication, we need to introduce gadgets in the tree that makes the tree asymmetric.
Introducing bandwidth constraints allows to control the number of nodes spawning in certain parts of the tree.

\subsection{Two Replicas without Bandwidth Constraints}\label{ap:tworep-ma}
\subsection{Two replicas without Multiple Assignment}\label{ap:tworep-ni}

\begin{enumerate}
  \item Two groups of nodes: for matching and for redundancy
\end{enumerate}