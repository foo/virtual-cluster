\subsection{Two replicas without Multiple Assignment}\label{ap:tworep-ni}


We now show that~$\RS(2)+\FP+\CC+\BW$ is even NP-hard without multiple
assignment.

\noindent \textbf{Construction.}

\emph{Chunk Types.}  We construct the following chunk types: For each
element~$e\in X\cup Y\cup Z$, we construct~$\deg(e)$ chunk types with
two replicas. Additionally, we construct
$\max\{3\cdot |T| + 3\cdot n + 1, \sum_e(2\cdot \deg(e)-1)\}$
chunk types called \emph{unique chunks}. We
refer to the set of unique chunks by~$U$.

\emph{The substrate network.}

\begin{enumerate}
  \item The physical network consists of three subtrees connected to
  the root: A {\MatchSubtree}, a {\CoverSubtree}, and a
  {\UnqSubtree}. In the {\MatchSubtree} we put $|T|$
  {\TripleGadgets} (remember that~$|T|$ is the number of triples in
  {\TDPM} instance). {\CoverSubtree} consist of~$n$ element gadgets.
  \item The {\UnqSubtree} consist of~$|U|$ leaves, and two middle nodes:
  a lower and an upper middle node. Note that this is different from $\RS(2)+\FP+\MA(4)$ NP-completeness proof, where {\UnqSubtree} was placed in the {\MatchSubtree}.
  \item \TripleGadget: For each triple, we create a subtree
  consisisting of four vertices: three leaves and one triple root.  We
  attach the root of the triple to the root of the matching subtree.
  \item \ElGadget: For each element~$e \in X\cup Y\cup Z$, we
  construct a subtree consisting of the root of the element (attached
  to the root of the cover subtree), and~$4\cdot(\deg(e)-1)+1$ leaves.
\end{enumerate}

\emph{Chunk placement.}
The chunks are placed as follows:
\begin{enumerate}
  \item \emph{Chunks in matching subtree:} For each triple~$\tau$ we put
  three chunks at the leaves of the corresponding \TripleGadget,
 ~$e_1(\tau), e_2(\tau), e_3(\tau)$.
  \item \emph{Chunks in unique subtree:} We place unique chunks~$U$ at
  the leaves of {\UnqSubtree}.
  \item \emph{Chunks in element gadget:} For each element
 ~$e\in X\cup Y\cup Z$, we place the chunks~$\tau(e)$ at the leaves of
  each Element Gadget.
\end{enumerate}


\emph{Bandwidth constraints.}
We use bandwidth constraints of the form
$\Band(k) := k\cdot(\numNodes - k)$. Namely, we set the bandwidth
constraints of an uplink of an {\ElGadget} for each element~$e$ to 
$\Band(\deg(e)-1)$, the bandwidth of an uplink of a~$\MatchSubtree$ to 
$\Band(n)$, and an uplink of a~$\CoverSubtree$ to 
$\Band(\sum_e (\deg(e)-1)$.

\emph{The threshold value and other properties of the instance.} We set the
cost threshold for any solution to the following value:

\begin{tiny}
\begin{align*}
  \Thr  & = 2\cdot (3\cdot \numNodes + \sum_e (\deg(e) - 1)) & \mbox{(over 2 hops)}\\
        & + 4\cdot (n\cdot (3 \cdot (3\cdot \numNodes - 3)) / 2) & \mbox{(over 4 hops in {\MatchSubtree})}\\
        & + 4\cdot (\sum_e((\deg(e) - 1)\cdot (\sum_{f\neq e} \deg(f) - 1)/2)) & \mbox{(over 4 hops in {\CoverSubtree})}\\
        & + 6\cdot (3\cdot \numNodes \cdot \sum_e(\deg(e) - 1)) & \mbox{(between {\MatchSubtree} and {\CoverSubtree})} \\
        & + |U|\cdot (|U|-1)/2 & \mbox{(inside {\UnqSubtree})} \\
        & + |U| \cdot (3\cdot n + \sum_e (\deg(e)-1) & \mbox{({\UnqSubtree} to other nodes)} \\
\end{align*}
\end{tiny}

  We
set~$\CostTrans$, the cost of chunk transportation to $\Thr+1$ (so that no chunk transportation happens in any feasible solution), 
$\CostCom = 1$, and we host only one node per machine. We set the
number of machines to spawn to:
$\numNodes := 3\cdot n + \sum_e (\deg(e)-1) + |U|$.
\\

\noindent \textbf{Properties of the substrate network.}
\begin{lemma}
  Assume we have a~$\RS(2)+\FP+\CC+\BW$ instance~$I$ with a subtree
 ~$T'$ with~$l$ leaves and the bandwidth capacity on uplink of~$T'$ is
 ~$\Band(k)$. Assume that no chunk transportation is allowed
  ($\CostTrans = \infty$, so every node must be collocated with the
  chunk it processes in every feasible solution), and~$\CostCom = 1$.
  Then in any feasible solution the number of nodes spawned in~$T$ (we
  name it~$s$) satisfies~$s \leq k \vee n-s\leq k$, and~$s \leq l$.
  \label{lem:bandwidth1}
\end{lemma}

\begin{proof}
It holds that ~$s\leq l$ as we cannot spawn more nodes than leaves.
  The bandwidth allocation on the uplink of~$T'$ is
 ~$uplink(s,T) := s\cdot (n - s)$, as no chunk transportation
  is allowed ($\CostTrans = \infty$), and every node in~$T$ has to
  communicate over~$T'$'s uplink with nodes spawned outside of
 ~$T'$. Therefore, in every feasible solution we have:
 ~$uplink(s, T') \leq \Band(k)$.  Let's define the remaining bandwidth
  on the uplink of $T'$~$\remainBw(s) := \Band(k)-uplink(s, T') = s^2 - s\cdot n -
  k^2+k\cdot n~$.
  Every feasible solution fulfills~$\remainBw(s) \geq 0$, which is true for
 ~$s \leq k \vee n-s\leq k$ (follows from the properties of the
  quadratic function).
\end{proof}


Next, we show how to precisely control the number of nodes in the
constructed subtree.

\begin{obs}
  In every feasible solution we have exactly~$|U|$ nodes spawned in a
  {\UnqSubtree} (no chunk transportation is allowed, and every chunk
  type must be processed).
  \label{obs:unq-full}
\end{obs}


\begin{lemma}
  Consider an instance~$I$ of~$\TDPM$. We construct the~$\RS(2)+\FP+\CC+\BW$ instance~$I'$
 as described above. Then we have that in~$I'$:
  \begin{enumerate}
    \item The number of nodes spawned in a {\MatchSubtree} is
   ~$3\cdot n$.
    \item The number of nodes spawned in a {\CoverSubtree} is
   ~$\sum_e(\deg(e)-1)$
  \end{enumerate}

  \label{lem:bandwidth2}
\end{lemma}

\begin{proof}
  From Observation~\ref{obs:unq-full} we know that we have~$|U|$ nodes
  in the {\UnqSubtree}. Let's refer to the number of nodes spawned in
  a {\MatchSubtree} by~$M$, and to the number of nodes spawned in
  {\CoverSubtree} by~$C$. By applying Lemma~\ref{lem:bandwidth1} to
  {\MatchSubtree}, we know that: $ M \leq 3\cdot n \vee M \geq |V| - 3\cdot n$.
  We observe that~$|V| - 3\cdot n$ is greater than the number of
  leaves in a {\MatchSubtree}.  By applying Lemma~\ref{lem:bandwidth1}
  to the {\CoverSubtree} we know that:~$ C \leq \sum_e(\deg(e)-1) \vee C$ $\geq |V| - \sum_e(\deg(e)-1)$.
  We observe that~$|V| - \sum_e(\deg(e)-1)$ is greater than the number
  of leaves in the {\CoverSubtree}.
  We also know that~$|V| = |U| + C + M$. Therefore, by the pigeon-hole principle
 ~$C = \sum_e(\deg(e)-1)$ and~$M = 3\cdot n$.
\end{proof}


\begin{lemma}
  Assume an instance~$I$ of~$\TDPM$. We construct the~$\RS(2)+\FP+\CC+\BW$ instance~$I'$
 as described above. Then we have that in~$I'$ the number of nodes spawned in Element Gadget of
  element~$e$ is~$\deg(e)-1$.
  \label{lem:bandwidth3}
\end{lemma}

\begin{proof}
  Let's call the number of nodes spawned in the Element Gadget of
  element~$e$ the $x_e$.  From Lemma~\ref{lem:bandwidth1}, we know that
 ~$x_e \leq \deg(e) - 1 \vee x_e \geq |V| - \deg(e) + 1$. We observe
  that~$|V| - \deg(e) + 1$ is greater than the number of leaves of the
  gadget, which is~$\deg(e)$.  From Lemma~\ref{lem:bandwidth2}, we know
  that the number of nodes spawned in the entire {\CoverSubtree} is
 ~$\sum_e (\deg(e)-1)$. Therefore, by the pigeon-hole principle, we have
  that~$x_e = \deg(e)-1$.
\end{proof}

From the above lemmas we know the precise number of nodes spawned in
certain parts of the tree. Feasible solutions only differ in 
the choice of the~$\deg(e) - 1$ out of~$\deg(e)$ chunks
in each Element Gadget, and the placement of nodes in the
{\MatchSubtree}.

%
% \begin{lemma}
% 
%   Assume we have a~$\RS(2)+\FP+\CC+\BW$ instance~$I$.  Assume that
%   no chunk transportation is allowed, and~$\CostCom = 1$.
%
%   Assume that we have a subtree~$T$ with subtries
%  ~$S_1, S_2, \ldots S_a$ attached to the root of~$T$. Assume that in
%   every feasible solution we have exactly~$Q$ machines in~$T$ (and
%   the rest~$\numNodes - Q$ machines outside of~$T$). The bandwidth
%   capacity on uplink of~$S_i$ is set to~$\Band(x_i)$, where
%  ~$\sum_i x_i = Q$. Then, in every feasible solution the number of
%   nodes spawned in~$S_i$ is~$x_i$.
%
%   \label{lem:bandwidth2}
% \end{lemma}
% 
%
% \begin{obs}
%   \label{obs:nodes-match-cover}
%   Using lemma \ref{lem:bandwidth2} with~$T$ being the whole tree,
%   and~$Q = \numNodes$ we conclude that
%   \begin{enumerate}
%     \item In every feasible solution there are~$3\cdot \numNodes$
%     nodes spawned in {\MatchSubtree}.
%     \item In every feasible solution there are~$\sum_e (\deg(e)-1)$
%     nodes spawned in {\CoverSubtree}.
%   \end{enumerate}
% \end{obs}
%
% \begin{obs}
%   \label{obs:deg-min-one}
%
%   Using lemma \ref{lem:bandwidth2} with~$T$ being the
%   {\CoverSubtree}, and~$Q = \sum_e (\deg(e)-1)$ we conclude that in
%   every feasible solution the number of nodes spawned in {\ElGadget}
%   of element~$e$ is exactly~$\deg(e)-1$.
%
%   Therefore, at least one of~$\deg(e)$ chunks that correspond to
%   element~$e$ was processed in {\MatchSubtree}.
% \end{obs}
%

Similar in spirit to the NP-completeness proof of~$\RS(2)+\MA(4)+\FP$,
we call the {\TripleGadget} active if it contains exactly three nodes. 
Similarly, we call the {\TripleGadget} inactive if it
does not contain spawned nodes, and \emph{partially active} if it
has one or two
spawned nodes.

\begin{lemma}
  Consider an~$\RS(2)+\FP+\CC+\BW$ instance~$I$.  Assume that 
  chunk transportation is not allowed, and~$\CostCom = 1$.
  In every feasible solution to~$I$, we have exactly~$n$ active
  {\TripleGadgets}.
  \label{lem:full-or-empty}
\end{lemma}

\begin{proof}
  Since~$I$ is feasible, we know that it has a solution~$\Sol$ of
  cost~$\leq \Thr$.
  By Lemma~\ref{lem:bandwidth2}, we know that there are
  exactly~$3\cdot n$ spawned nodes in the {\MatchSubtree}. Therefore, by
  the pigeon-hole principle, we know that we have at most~$n$
  active {\TripleGadgets}. It remains to show that there
  are no partially active {\TripleGadgets} in the solution of cost
 ~$\leq \Thr$.
  Using Lemma~\ref{lem:bandwidth3}, 
  we conclude that the communication cost of
  nodes in the {\CoverSubtree} is the same for every feasible solution
  (let's name that cost~$P$). We also know that the communication cost
  between nodes in {\CoverSubtree} and {\MatchSubtree} is the same for
  every feasible solution (let's name it~$Q$). Let's call the
  would-be cost of communication in the {\MatchSubtree}, if there were
  exactly~$n$ active gadgets,~$R$.
  The threshold value was chosen so that~$\Thr = P+Q+R$. If we have at least one partially active
  gadget, then the cost of communication in {\MatchSubtree} is greater
  than~$R$, because we increase the number of 4-hop communications by
  at least one per each partially active gadget in comparison to a solution
  where we have exactly~$n$ active gadgets.
\end{proof}

\noindent \textbf{The reduction.}

\begin{theorem}
 ~$\RS(2)+\FP+\CC+\BW$ is NP-hard.
\end{theorem}

\begin{proof}
  Let's take an instance~$I$ of~$\TDPM$ and construct an instance~$I'$
  of~$\RS(2)+\FP+\CC+\BW$ in the way described above.  We show that
 ~$I'$ has solution of cost~$\leq \Thr$ if and only if~$I \in \TDPM$
  (there exists a perfect 3D matching).

  ($\Leftarrow$) Let's take any feasible solution~$\Sol$ to~$I$ and
  produce a solution~$\Sol'$ to~$I'$. We show that the cost of~$\Sol'$ is
  indeed~$\leq \Thr$.
  For each triple~$t_1,\ldots, t_n$ in~$\Sol$, we put~$3$ nodes at
  leaves of triple gadgets corresponding to those triples.  In each
  element gadget (that corresponds to element~$e$), we put~$\deg(e)-1$
  nodes. In each element gadget there is only one leaf without the
  node placed in it: this node contains the chunk replica that is
  processed in the {\MatchSubtree}.
  It is easy to see that~$\Sol'$ has cost exactly~$\Thr$ and no
  bandwidth constraint is violated. Each chunk type is processed.

  ($\Rightarrow$) Let's take any feasible solution~$\Sol'$ to~$I'$ and
  produce a solution~$\Sol$ to~$I$ by taking triples that correspond
  to active triple gadgets. Using Lemma~\ref{lem:full-or-empty}, we
  conclude that there are exactly~$n$ active triple gadgets. By
  feasibility of~$\Sol'$, we know that each chunk type is
  processed. From Lemma~\ref{lem:bandwidth3}, we know that out
  of~$\deg(e)$ chunk types that correspond to~$x\in A\cup B\cup C$,
  exactly one is processed in the {\MatchSubtree}, and therefore each
  element of~$A\cup B\cup C$ is covered.
\end{proof}

