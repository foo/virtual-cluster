\section{(Journal version) NP-completeness of 2-replica variant}

\subsection{Intro}

\begin{enumerate}
  \item Needed properties: FP, RS(2), MA(4)
  \item Eliminated any need for bandwidth constraints, node interconnect and idle machines
\end{enumerate}

\subsection{Notation}

(3DPM) We call $deg(e)$ the number of triples that contain element
$e\in A\cup B\cup C$. We call $n=|A|$; please note that
$n=|A|=|B|=|C|$. We call $t$ the number of triples.

\subsection{The construction}

Let's take any instance $I$ of 3D Perfect Matching and create a VCEMB
instance $I'$ in the following way:

\begin{enumerate}
  \item (core chunk types) for each element $e \in A\cup B\cup C$ we create
  $deg(e)+a$ chunk types, where $a\in{0,1,2}$ is to make $deg(e)-1+a$
  divisible by 3. We those chunk types $core(e)$ (both corresponding to
  degree and those additional).
  \item (unique chunk types) We create additional chunk types which
  will have only one replica. We create $n$ chunk types and we call
  this set $mtch$. For each element $e\in A\cup B\cup C$ we create $core(e)/3$
  chunk types called $unqCore(e)$.
  \item (convention for core chunk types naming) Let's say that $e$ is in triples
  $t_1,t_2,\ldots,t_{deg(e)}$; then $core(e) = \{ e^{t_1}, e^{t_2},
  \ldots, e^{t_{deg(e)}} \} \cup ADDITIONAL$, where $ADDITIONAL$ are
  the chunk types that were added to make $deg(e)-1+a$ divisible by
  3.
  \item (tree) physical network consists of two parts: Matching Subtree
  and Cover Subtree. Matching Subtree consists of two parts: $t$ Triple
  Subtries and Unique Subtree. Cover Subtree consists of $|A\cup B\cup
  C|$ Element Subtries.
  \item (Triple Subtree) For each triple we create a subtree consisisting of
  4 vertices: three leaves and one root of the triple. We attach the
  root of the triple to root of Matching Subtree.
  \item (Unique Subtree) We create $n$ leaves and $n$ middle nodes and
  connect them to root of Matching Subtree. Note that we create middle
  nodes not only to keep the tree balanced, but also to keep 
  leaves of Unique Subtree far from other leaves of Unique Subtree.
  \item (chunks in Triples Subtree) For each triple
  $t = \langle a_i, b_j, c_k \rangle$ we put three chunks in leaves of
  the corresponding gadget: $a_i^{t}, b_j^{t}, c_k^{t}$. This
  way there are no two replicas of the same chunk type in the triples
  subtree.
  \item (chunks in Unique Subtree) We put chunks $mtch$ in leaves of
  Unique Subtree.
  \item (Element Subtree) For each element $e \in A\cup B\cup C$ we construct the
  subtree consisting of a root of the element (attached to the root of
  Cover Subtree), and with $|core(e)| + |unqCore(e)|$ leaves. We
  \item (chunks in Element Subtree) We place chunks $core(e) \cup
  unqCore(e)$ in leaves of each Element Subtree.
  Please note that no two replicas of the same chunk
  is placed in Cover Subtree.
  \item (multiple assignment) We set the number of processed chunks by
  each machine to 4.
  \item (number of machines) We allow to spawn $numVMs := n + \sum_{e\in A\cup
    B\cup C}|core(e)|/3$ machines.
  \item (threshold) $Thr := n\cdot (0 + 2 + 2 + 4) + \sum_{e\in A\cup
    B\cup C}|core(e)|/3 \cdot (0 + 2 + 2 + 2)$
\end{enumerate}

\subsection{The reduction -- constructing VCEMB solution from 3DPM
  solution}

\begin{enumerate}
  \item We put $n$ VMs in any leaf of such Triple Subtries that were
  chosen in 3DPM solution. For each such VM we match to it 3 chunks
  that are put in its Triple Subtree (it is sitting on one of them),
  and we match it to any unmatched chunk in Unique Subtree.
  \item We put $|core(e)|/3$ VMs in each Element Subtree. We match all remaining
  chunks from this Element Subtree to those VMs in any possible
  way.
  \item Such a solution has cost $\leq Thr$ (easy to see by
  summing costs of transporting chunks to each VM)
  \item Feasibility of the constructed solution. Each chunk type was covered
\end{enumerate}

\subsection{The reduction -- constructing 3DPM solution from VCEMB
  solution}

\begin{theorem}
  Let's take any feasible solution $S$ to the instance $I$ of VCEMB
  (as constructed above). If cost of $S$ is at most $Thr$, then no VM
  is spawned in Unique Subtree.
  \label{th:no-unique}
\end{theorem}

\begin{proof}
  Let's assume the contrary, that in feasible solution $S$ at least
  one VM spawned in Unique
  Subtree. We show that in this case, the cost of solution $S$ is
  greater than $Thr$. 

  Let's name $l$ the number of VMs that was spawned in $S$ in Unique
  Subtree. We know that $1 \leq l \leq t$. We denote the set of VMs
  spawned in Unique Subtree as $U_{VM}$.

In $S$ we have exactly
  $4 \cdot numVMs$ chunk transportations, some incurring cost $0, 2,
  4$ or $6$ (tree has edge-height of $3$). At most $numVMs$
  transportations are of cost $0$. Note that leaves of Unique Subtree
  are separated from other leaves of the tree by at least $4$ edges. 

  Cost of chunk transportation to VM $v \in U_{VM}$ is at least $0 + 4
  + 4 + 4$. The chunks in Unique Subtree are unique, therefore we need
  to transport $t - l$ chunks to machines outside of the Unique
  Subtree, incurring cost $4$ for each of the chunk.

  We lower-bound the cost of each of remaining transportations as $2$.

Summing the total cost, we have:

$cost(S) \geq costEstim(l) = 0 \cdot numVMs + l \cdot 12 + (t-l)\cdot 4 +
(4\cdot numVMs - numVMs - 4\cdot l - (t-l))\cdot 2$

It can be verified that for each $l$ we have that $costEstim(l) > Thr$. To
do so, we first show that $costEstim(l+1) \geq costEstim(l)$, for
$1\leq l \leq t-1$. Then we show that $costEstim(1) \geq Thr$.
\end{proof}

\begin{theorem}
  Let's take any feasible solution $S$ to the instance $I$ of VCEMB
  (as constructed above). If cost of $S$ is at most $Thr$, then
  exactly $\sum_{e\in A\cup B\cup C}|core(e)|/3$ VMs spawn in Cover
  Subtree and remaining $n$ VMs spawn in Matching Subtree
\end{theorem}

\begin{proof}
  Let's assume the contrary, that is that the number of VMs $M_{VM}$ spawned in
  the Matching Subtree, and $|M_{VM}| \neq n$. We denote
  $l=|M_{VM}|$. First, we use the theorem
  \ref{th:no-unique} to restrict the placement of VMs in Unique Subtree. Then we consider cases:
  \begin{enumerate}
    \item $l \leq n$. In this case there are at least $u := 4 \cdot (n-l)$
      chunks in the Matching Subtree that are not processed in the
      Matching Subtree, each incurring transportation cost of $6$. We
      lower-bound the cost of $S$ in the following way:

$cost(S) \geq costEstim2(u) = 0\cdot numVMs + 4\cdot t + (4\cdot
numVms - numVMs - t)\cdot 2$. 

It can be verified that for each $u$ we have that $costEstim2(S) > Thr$. To
do so, we first show that $costEstim2(u+1) \geq costEstim2(u)$, for
$1\leq u \leq n-1$. Then we show that $costEstim2(1) \geq Thr$.

    \item More than $n$ VMs spawned in Matching Subtree. Sketch of the
      proof: the same as in previous enumerate: we count how many
      chunks from Elements Subtree has to be processed outside of it
      (incurring cost $6$);
      we lower-bound the cost of remaining transportations as: $4$
      cost of $t$ chunks in Unique Subtree, and $2$ for the rest of them.
  \end{enumerate}
\end{proof}

Using above properties, we construct the 3DPM solution by taking each
triple that has one VM in its corresponding Triple Subtree.

\begin{enumerate}
\item TODO: Pictures of the network graph in MA proof
\item TODO: Pictures of the network graph in NI proof
\item TODO: Description of the construction in NI proof
\item TODO: Proof the lemmas in MA proof
\item TODO: Formulate and proof the bandwidth lemma in NI proof
\end{enumerate}