In order to show NP-completeness of our problem we will perform
reduction from 3-SAT. Let's take any formula $\Phi$ in 3-CNF. First,
transform $\Phi$ to equivalent formula $\Psi$ the following way:

\begin{itemize}
\item let's assign every clause its unique number from 1 to number of
  clauses
\item for every occurence of variable $v$ we replace it to $v_{number
    of clauses}$
\item for every variable $v$ we add set of clauses in form $v_i \iff v_{i+1}$
\end{itemize}

Let's construct our tree out of gadgets for every variable connected
to tree root. The gadget is constructed as follows:

\begin{itemize}
\item i-th occurence of $v$ we create two leaves labeled $v_i$ and
  $\neg v_i$
\item we connect positive leaves in one subtree with root vertex called ``positive'' and negative leaves in
  another subtree with root vertex called ``negative''
\item we connect ``positive'' and ``negative'' into one subtree with
  root vertex called $v$
\item we set bandwidth constraints between $v_i$ and ``positive'' to
  $vms - 1$
\item we set bandwidth constraints between $\neg v_i$ and ``negative''
  to $vms - 1$
\item we set bandwidth constraints between ``negative'' and $v$ to
  $vms - \binom{number of occurences of v}{2}$
\item we set bandwidth constraints between ``positive'' and $v$ to
  $vms - \binom{number of occurences of v}{2}$
\item we set bandwidth constraints between $v$ and root to
  $vms - \binom{number of occurences of v}{2}$
\end{itemize}

For each clause we place 3 chunks in 3 leaves labeled as literals in
this clause.

We set number of virtual machines to place to be $3 \cdot number of
clauses$.

We set chunk transfer cost to be any positive integer, let's say 1.
We set VM communication transfer cost to be 1.

Let's define $C$ as VM communication cost when all machines are in
positive literals.

This ends our construction. Now we need to decide $\Phi$. We do it the
following way:

\begin{itemize}
\item if there is no solution that doesn't exceed bandwidth, we
  answer: $\Phi$ is unsatisfiable
\item if minimal solution has value larger than $C$, we answer: $\Phi$
  is unsatisfiable
\item if minimal solution has value $C$, we answer: $\Phi$ is satisfiable
\end{itemize}

Proof of correctness.

First, we will proove that if $\Phi$ is satisfiable then our VC
instance has solution with cost equal to $C$. Let's take any valuation $F$
that satisfies $\Phi$. For every variable $v$ that is set to true we
place virtual machines in every vertex with label $v_i$. For every
variable $v$ that is set to false we place virtual machines in every
vertex with label $\neg v_i$. For every virtual machine we either have
a chunk in the same vertex - then we match the VM with the chunk - or
we do not have a chunk in the same vertex - then we do not match the
VM with any chunk. We need to proove two claims:
\begin{itemize}
\item This placement of virtual machines does not exceed the bandwidth
\item This placement of virtual machines matches every chunk
\end{itemize}

TODO: prove claims above.

Second, we will proove that if our VC instance has solution with cost
equal to $C$ than $\Phi$ is satisifiable. Let's create valuation $F$
in the following way:

\begin{itemize}
\item if $v_1$ has machine in it then $F(v) = true$
\item if $\neg v_1$ has machine in it then $F(v) = false$
\end{itemize}

In order to prove that $F$ satisfies $\Phi$ we will need to proove
following claim. If $v_1$ has virtual machine then for all $i$ we have
that $v_i$ has virtual machine and for all $i$ we have that $\neg v_i$
does not have a virtual machine. Also, if $\neg v_1$ has virtual
machine then for all $i$ we have that $\neg v_i$ has virtual machine
in it and for all $i$ we have that $v_i$ does not have a virtual
machine. To do: prove this claim.