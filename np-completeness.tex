We have seen that even problems with multiple dimensions of
flexibility, e.g., for the joint optimization of flexible node
placement and replica selection, and even under capacity constraints
(our problem ($\FP+\CC+\MA+\BW$), can be solved in polynomial time.
This section now points out fundamental limitations in terms of
computational tractability. In particular, we will show that problems
become NP-hard if multiple replicas have to be assigned to a node
($\FP+\RS+\MA$ is proved NP-hard in Section~\ref{ssec:fprsma}) or if
inter-connects have to be established ($\FP+\RS+\CC$ is proved NP-hard
in Section~\ref{ssec:fprscc}); both results hold even in uncapacitated
networks, and even in fat-trees consisting of two levels only (e.g.,
in a datacenter \emph{pod}). Finally, we show that some problems are
NP-hard even if the number of replicas is only two
(Section~\ref{ssec:two}).

\subsection{Introduction to 3D Perfect Matching}

Common introduction for both proofs.

\begin{enumerate}
\item links: \url{http://www.nada.kth.se/~viggo/wwwcompendium/node143.html}
\item links: \url{https://en.wikipedia.org/wiki/3-dimensional_matching}
\item describe the Karp's variant of the problem, where each of X, Y, Z have the same size
\item intersection of bounded size set cover and exact cover
\item describe what an instance consists of
\end{enumerate}

The reduction is from three-dimensional matchings, a generalization of bipartite matchings
to 3-uniform hypergraphs, henceforth called
$\TDM$.~\cite{3dmatch}

$\TDM$ is defined as follows. We are given three finite and disjoint sets $X$, $Y$, and $Z$,
as well as a subset of triples $T\subset X \times Y \times Z$.
Set $M \subseteq T$ is a 3-dimensional matching if and only if,
for any two distinct triples $(x_1, y_1, z_1) \in M$ and $(x_2, y_2, z_2) \in M$,
it holds that $x_1\neq x_2$, $y_1\neq y_2$, and $z_1\neq z_2$. Our goal is to
maximize the cardinality of $M$.


\subsection{Multi-Assignments are hard ($\FP+\RS+\MA$)}\label{ssec:fprsma}

First, we show the hardness of a problem variant without inter-connect,
but where nodes are assigned multiple chunks, namely the $\FP+\RS+\MA$
problem.
\textbf{Construction.}
Given an instance $I$ of $\TDM$, we construct an instance $I'$ of
$\FP+\RS+\MA$ as follows:
\begin{itemize}
\item We construct a tree consisting of gadgets, where each gadget corresponds to a triple.
\item Chunks correspond to elements in $X$, $Y$ and $Z$
\item $|X| = |Y| = |Z| = k$
\item we set $\CostTrans=1$
\item we set m to 3
\item we set number of VMs to $k$
\item we set $\Thr=2 \cdot 2 \cdot k$
\item each gadget consists of 3 leaves and parent; in leaves there are
  chunks that correspond to elements of a triple
\end{itemize}

FIXME: The construction is illustrated in Figure~\ref{fig:fprsma}.

\textbf{Correctness.}
We can now show the computational hardness.
\begin{theorem}
$\FP+\RS+\MA$ is NP-hard.
\end{theorem}
\begin{proof}
Let $I$ be an instance of $\TDM$ and let $I'$ be an instance of
$\FP+\RS+\MA$ constructed as described above.
We prove that $I'$ has solution of cost $\leq \Thr$ if ($\Rightarrow$) and only if
($\Leftarrow$)
$I$ has a matching of size $k$.

($\Rightarrow$) Let us take a solution to $\TDM$. We spawn VM in every
gadget that corresponds to chosen triples. We match every chunk in a
gadget to machine in this gadget (only for chosen ones). Solution has
cost exactly $\Thr$.

($\Leftarrow$) Let's take solution to $VC$ instance of cost $\leq \Thr$. We
chose triples that correspond to gadgets where were VMs. Everything
was processed, therefore every element of X,Y and Z is matched.
\end{proof}


\subsection{Inter-connects are hard ($\FP+\RS+\CC$) (reduction from 3D Perfect Matching)}\label{ssec:fprscc}


Next, we prove that the joint optimization of node placement and replica selection
is NP-hard if an inter-connect has to be established between virtual machines.
In our terminology, this is the $\FP+\RS+\CC$ problem.


\textbf{Construction.}
Given an instance of $\TDM$, we construct an instance of a
$\FP+\RS+\CC$ problem as follows. For each element
in the universe $X \cup Y \cup Z$, we construct a chunk; and for each
tripple $T_i$, we construct a gadget that contains
three replicas of chunks corresponding to the elements in the subset.
We connect the gadgets to the root, separating nodes from different tripples by 4 hops.

Concretely, let $I$ be an instance of $\TSC$. We will create an instance $I'$
for $\FP+\RS+\CC$ as follows:
\begin{itemize}
\item We set the access cost $\CostTrans$ to a replica to a high value $W$. This will force
nodes to be collocated with the replica.
(For now, we can assume that $W=\infty$; a lower and sufficient bound will be given
in the appendix.)
\item The communication cost in the inter-connect is set to $\CostCom = 1$.
\item The number of nodes (virtual machines) is $\Vms = 3 \cdot k$. TODO: check again
\item We use a threshold $\Thr =  3 \cdot k + 3 \cdot 3 \cdot 2 \cdot (k - 1)$. TODO: check again
\end{itemize}

We construct a height-2 substrate tree
as follows. For each $T_i$ we construct a gadget
consisting of an inner node (a router) and three leaves. Every gadget
contains three chunks, corresponding to the elements of $T_i$.

FIXME: The construction is illustrated in Figure~\ref{fig:fprscc}.

\textbf{Proof of correctness.}
Intuitively, in order to minimize embedding costs,
nodes should be placed on near-by replicas. We use the following
helper lemma.
\begin{lemma}\label{lemma:helper}
In every valid solution $\Sol$ of $I'$ of cost $\leq \Thr$, each gadget
falls in one of two categories:
$k$ gadgets have exactly
$3$ nodes, and $n-k$ gadgets remain empty.
\end{lemma}
\begin{proof}
Since $W=\infty$, nodes will always be placed
directly on chunks (the access network cost is zero).
Moreover, since
$\Sol$ is valid, $3 \cdot k$ nodes are mapped
directly to the different chunk locations.
Now, consider any pair of nodes communicating over the
inter-connect; due to our construction, the communication cost
for each such pair is either
2 hops (if they belong to the same gadget) or 4 hops (if they belong
to different gadgets).
The lemma then follows from the observation that $\Thr$
is chosen such that it is never possible to distribute nodes
among more than $k$ gadgets, and that it is always strictly better to
have exactly 0 or 3 nodes per gadget, than any alternative distribution.
\end{proof}

\begin{theorem}
$\FP+\RS+\CC$ is NP-hard.
\end{theorem}
\begin{proof}
Let $I$ be an instance of $\TDM$ and let $I'$ be an instance of
$\FP+\RS+\CC$ constructed as described above.
We prove that $I'$ has solution of cost $\leq \Thr$ if ($\Rightarrow$) and only if
($\Leftarrow$)
$I$ has a solution.

($\Rightarrow$) In order to compute a solution
for $I'$ given a solution for $I$, we proceed as follows.
Given a covering set of tripples $S = \{T_1, T_2, \ldots, T_k\}$, we place three nodes in each gadget that
corresponds to every tripple of $S$. Chunks are matched to VMs that sit on top of them.

The solution has the following cost:
(1) the communication cost inside a gadget is $2 \cdot {3 \choose 2}$,
  as every pair contributes two hops;
  (2) the communication cost from each gadget to all other gadgets is $4
  \cdot 3 \cdot 3 \cdot (k - 1) / 2$, where the factor $2$ is
  for the
  communication over $4$ hops, the factor $3$
  corresponds to the number of nodes per gadget, and
  $3 \cdot (k-1)$ is the number of nodes in remote gadgets;
  as we count each pair twice, we need to divide by two in the end.
Summing up over all $k$ gadgets, we get exactly $\Thr$.

($\Leftarrow$) Given a solution for $I'$,
we can exploit Lemma~\ref{lemma:helper} to construct a solution for $I$.
We know that in any solution of cost at most $\Thr$,
$k$ gadgets contain exactly 3 nodes. These gadgets correspond to a valid
set cover of size $k$: every
chunk and hence element in the universe, is covered.
\end{proof}

