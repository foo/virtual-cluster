\subsection{Dynamic programming based solutions}

Dynamic programming is an algorithmic paradigm, which can be exploited in 
order to solve instances of $\Problem$, which contain the properties $fp$ + 
$cv$ + $ma$ and $bw$. This section will describes an dynamic programming based 
algorithm which solves $fp$ + $cv$ + $bw$, and points out how to extend or 
reduce it to other combinations of properties.

TODO: if we ever want to put 2 replica NP-completeness proof, we need to 
introduce idle machines

TODO: here we need to explicitly say what we mean by mixing MA and idle machines; just as a reminder, we decided to force the machines to be full or empty

%\subsection{new}

\newcommand{\Opt}{\ensuremath{Opt}}
\newcommand{\Children}{\ensuremath{children}}
%\newcommand{\Cost}{\ensuremath{\textsc{cost}}}

\newcommand{\Uplink}{\ensuremath{\textsc{uplink}}}
\newcommand{\ChunkCount}{\ensuremath{\textsc{ChunkCount}}}


To compute a solution $\Sol$ for an instance of $\Problem$, we compute an 
integer $\ChunkCount_\SubstrateNode$ and an array 
$\Opt_\SubstrateNode$ on each node $\SubstrateNode \in 
\SubstrateNodes$. $\ChunkCount_\SubstrateNode$ contains the number of chunks at 
or below 
$\SubstrateNode$ and $\Opt_\SubstrateNode[i]~i\in \{0,\dots,\Vms\}$ contains the 
total cost ($\CostCom$ + $\CostTrans$) for placing $i$ VMs at or beneath 
$\SubstrateNode$. These costs account for bandwidth costs on 
$\Uplink(\SubstrateNode)$. To denote that it is impossible to place $i$ VMs at 
or beneath $\SubstrateNode$, we set $\Opt_\SubstrateNode[i] = \infty$.


\newcommand{\NodesToProcess}{\ensuremath{\textsc{nodesToProcess}}}

\begin{algorithm}[tbhp]
\DontPrintSemicolon % Some LaTeX compilers require you to use
%\dontprintsemicolon instead
\SetAlgoNoEnd
\KwIn{$\SubstrateNode \in \SubstrateNodes$}
\vspace{6pt}
\While{$|\Children(\SubstrateNode) > 2|$}{
  \textbf{set} $\SubstrateNodes \gets \SubstrateNodes \cup 
\{\SubstrateNode^*\}$\;
  \textbf{~~with }$\Capacity(\SubstrateNode^*) \gets 0$
  \textbf{ and }$\Cost(\SubstrateNode^*) \gets  1$\;
  \textbf{choose any }$\SubstrateNode'$\textbf{ from 
}$\Children(\SubstrateNode)$\; 
  $cap \gets \Capacity((\SubstrateNode, \SubstrateNode'))$\;
  $\SubstrateEdges \gets \SubstrateEdges \setminus \{(\SubstrateNode, 
\SubstrateNode')\}$\;
  \textbf{set} $\SubstrateEdges \gets \SubstrateEdges \cup \{(\SubstrateNode^*, 
\SubstrateNode')\}$\;
  \textbf{~~with }$\Capacity((\SubstrateNode^*, \SubstrateNode')) i\gets cap$
  \textbf{ and }$\Cost((\SubstrateNode^*,\SubstrateNode')) \gets  1$\;
  \textbf{set} $\SubstrateNodes \gets \SubstrateNodes \cup \{(\SubstrateNode^*, 
\SubstrateNode)\}$\;
  \textbf{~~with }$\Capacity((\SubstrateNode^*, \SubstrateNode)) i\gets \infty$
  \textbf{ and }$\Cost((\SubstrateNode^*, \SubstrateNode)) \gets  0$\;
  $binarize(\SubstrateNode')$\;
}
\For{\textbf{each } $\SubstrateNode' \in \Children(\SubstrateNode)$}{
  $binarize(\SubstrateNode')$\;
}

\caption{$binarize(\SubstrateNode \in \SubstrateNodes)$}
\label{algo:binarization}
\end{algorithm}
\vspace{-1.5em}

Our dynamic program works on bianry trees. Algorithm~\ref{algo:binarization} 
shows the algorithm to recursively binarize a (sub-)tree with the root $v$:
%To binarize tree $\Tree$, we start by processing each vertex $v \in 
%\SubstrateNodes$ in the following root to leaf manner: 
While $v$ has more than 
$2$ children, we choose an arbitrary node $v' \in \Children(v)$ 
and create a new virtual node $v^*$. We remove the edge $e = (v, v')$ which 
connected $v$ to it's child $v'$ from $\SubstrateEdges$. Subsequently we create 
a new edge, which connects the 
child to the new node $e^*_1 = (v', v^*)$, set it's 
$\Cost(e^*_1) = 1$ and it's capacity to the value of the removed edge $e$ 
($\Capacity(e^*_1) = \Capacity(e)$). Additionally we gerate an edge $e^*_2 = 
(v^*, v)$. $\Capacity(e^*_2)$ is $\infty$ and $\Cost{e^*_2}$ is set to $0$.
Subsequently we process the subtree with the root $v'$. If the current node has 
two or less children $v$ does not violate the binary constraint, and hence the 
binarization process of $v$ is finished. However, it is neccessary to 
recursively binarize the subtrees with roots $v'_1, v'_2 \in \Children(v)$, if 
they exists.
%If $|\Children(v)| > 2$, we choose an arbitrary node $v' \in \Children(v)$ 
%and create a new virtual node $v^*$. We create a new edge $e^*_1$, set it's 
%cost to $1$ and it's capacity to $\Bandwidth(e)$.

After the binarziation of $\Tree$ and the initialization of 
$\OPT_\SubstrateNode \forall \SubstrateNode \in \SubstrateNodes$ 
(Line~\carlo{todo}), we can compute $\OPT_\SubstrateNode$ in a bottom up 
manner. Since only one VM can be mapped to a leaf, $\OPT_\SubstrateNode[i] = 
\infty \forall i \in \{2,..,\Vms\}$. The values of 
$\OPT_\SubstrateNode[i]$ for $i \in \{0,1\}$, depend on the presence of a Chunk 
at $\SubstrateNode$. If there is a chunk at $\SubstrateNode$, and we decide to 
place $0$ VMs at or beneath $\SubstrateNode$, we will have to process the chunks 
data at a different location - which will inflict $\CostTrans$ bandwidth costs 
on $\Uplink(\SubstrateNode)$. If we decide to place a VM at $\SubstrateNode$, 
we can assign the VM to the chunk. This will require the VM to communicate with 
the other $\Vms -1$ VMs, which will inflict communication costs of $(\Vms -1) 
\cdot \CostCom$. If there is no chunk at $\SubstrateNode$ 
$\OPT_\SubstrateNode[0] = 0$, since there will be no traffic to 
$\SubstrateNode$ if we do not place a VM at $\SubstrateNode$. If we decide to 
place a VM at $\SubstrateNode$, this VM will communicate with the other VMs 
($(\Vms -1) \cdot \CostCom$) and it will have to fetch the data of a chunk from 
a different location ($\CostTrans$), hence $\OPT_\SubstrateNode[1] = (\Vms -1 
)\cdot \CostCom  + \CostTrans$. If the overall banwdith costs exceed 
$\Capacity(\Uplink(\SubstrateNode))$, $\Opt_\SubstrateNode$ is set to $\infty$.
$\ChunkCount_\SubstrateNode$ is obviously set to 
$1$ if a 
chunk is present at $\SubstrateNode$ and otherwise to $0$.

\newcommand{\SumIndex}{\ensuremath{n}}
\begin{algorithm}[tbhp]
\DontPrintSemicolon % Some LaTeX compilers require you to use
%%\dontprintsemicolon instead
\SetAlgoNoEnd
\KwIn{$\Opt_{\SubstrateNode_l} , 
\Opt_{\SubstrateNode_r}, 
\ChunkCount_{\SubstrateNode_l},\ChunkCount_{\SubstrateNode_r} $}
$\ChunkCount_\SubstrateNode = \ChunkCount_{\SubstrateNode_l} + 
\ChunkCount_{\SubstrateNode_r}$\;
\For{$\SumIndex \in \{0,\dots,\Vms\}$}{
  \For{$i \in \{0,\dots,\SumIndex\}$}{
      \If{$\Opt_\SubstrateNode[\SumIndex] > \Opt_{\SubstrateNode_l}[i] + 
\Opt_{\SubstrateNode_r}[\SumIndex - i]$}{
	$\Opt_\SubstrateNode[\SumIndex] \gets \Opt_{\SubstrateNode_l}[i] + 
\Opt_{\SubstrateNode_r}[\SumIndex - i]$\;
    }
  }
  
 $bw \gets (\Vms - 
\SumIndex) \cdot \SumIndex \cdot \CostCom +   |i - 
\ChunkCount_\SubstrateNode| \cdot \CostTrans$\; 
  \eIf{$bw \leq \Capacity(\Uplink(v))$}{
    $\Opt_\SubstrateNode[\SumIndex] \gets \Opt_\SubstrateNode[\SumIndex] + bw$\;
  }{
    $\Opt_\SubstrateNode[\SumIndex] \gets \infty$\;
  }
}
%
\caption{$aggregate(\SubstrateNode \in \SubstrateNodes)$}
\label{algo:dynAggregation}
\end{algorithm}

Ongoing we can compute $\OPT_\SubstrateNode$ for any substrate node 
$\SubstrateNode$, such that both children $\SubstrateNode_l$ and 
$\SubstrateNode_r$ of $\SubstrateNode$ have allready been processed. 
Algorithm~\ref{algo:dynAggregation} shows this process. At first we update the 
compute the chunks at or below $\SubstrateNode$ - which is the sum of the 
chunks below the children of $\SubstrateNode$, since chunks can only be placed 
on leaves (Line~\carlo{TODO}). Keeping in mind, that $\Capacity(\SubstrateNode) 
= 0$ for all non-leaf nodes $\SubstrateNode$, it becomes obvious, that for 
placing $\SumIndex$ VMs at or below $\SubstrateNode$, $i\in\{0,\dots, 
\SumIndex\}$ VMs have to be placed below one child of $\SubstrateNode$ and the 
other $\SumIndex - i$ children have to be placed below the other child of 
$\SubstrateNode$. The costs for placing $\SumIndex$ can be divided in two 
parts: Costs of placing $i$ VMs beneath one and $\SumIndex - i$ VMs beneath the 
other child of $\SubstrateNode$ and the banwdith costs on 
$\Uplink(\SubstrateNode)$. The costs for placing VMs beneath the children, is 
computed by checking all possible combinations (Lines~\carlo{TODO}). The 
bandwidth costs consist of two components, which are both independent of below 
which of the two children, the VMs are placed. The first component, is the 
communication costs (Line~\carlo{TODO}). Since $\SumIndex$ VMs are placed below 
$\SubstrateNode$ they will all communicate with the remaining $\Vms - \SumIndex$ 
VMs, which are not hosted beneath $\SubstrateNode$, generating a total banwdith 
consumption of $(\Vms - \SumIndex)\cdot \SumIndex \cdot \CostCom$. The 
second component are the transport costs (Line~\carlo{TODO}) which occur, if 
the number of chunks below $\SubstrateNode$ is not equal to the number of VMs 
below $\SubstrateNode$. If there are less VMs than chunks, we will have to 
transfer the data of the chunks to VMs which are not hosted below 
$\SubstrateNode$. If there are more VMs than chunks, we will have to transfer 
data from chunks which are not located below $\SubstrateNode$ to VMs which are 
hosted below $\SubstrateNode$. If the bandwidth costs for a specific number of 
VMs $\SumIndex$ exceed $\Capacity(\Uplink(\SubstrateNode))$, we set 
$\Opt_\SubstrateNode[\SumIndex] = \infty$, otherwise we add the bandwidth costs 
on the uplink to the costs for placing $i$ and $\Vms - i$ in the children of 
$\SubstrateNode$.

After computing $\Opt_\SubstrateNode~\forall 
\SubstrateNode \in \SubstrateNodes$ in the described bottom-up manner, the 
state in the root of the tree indicates, whether a solution for this instance 
of $\Problem$ exists. In case $\Opt_{root(\Tree)}[\Vms] = \infty$, no solutions 
exists. This will occur if the uplinks of the subtrees do not have sufficient 
spare capacity to satisfy the request, or the number of requested VMs is higher, 
than the available slots in the substrate. An actual embedding of cost 
$\Opt_{root(\Tree)}[\Vms]$ can be obtained by keeping track of the intermediate 
solutions for each array $\Opt_\SubstrateNode$ by indicating the number of 
nodes mapped in the subtree of the children.

The above described procedure can easily be extended to solve problem 
instances, which have the $MA$ property. Line~\carlo{TODO} of 
Algorithm~\ref{algo:dynAggregation} has to be modifed to
$$bw \gets (\Vms - \SumIndex) \cdot \SumIndex \cdot \CostCom +   
|\textbf{\MaFactor}\cdot i - \ChunkCount_\SubstrateNode| \cdot \CostTrans$$ 
in order to account for the fact, that each VM processes $\MaFactor$ many 
chunks. The same modification has to be added to the computation on the leaf 
nodes.

\begin{comment}
\subsection{old}

TODO:
\begin{enumerate}
  \item unify variable names $n$, $N$, etc
  \item formulate and prove local matching lemma
  \item take care of idle VMs! Parametrize cost by 3 variables instead
    of 2
\end{enumerate}


Let's start by transforming our tree to binary tree with arbitrary
depth. We also introduce weights on edges (either $0$ or $1$). The
strategy we use is to clone every vertex $|children(v)| - 2$ times,
placing subsequent clones as right son of the previous one and placing
subsequent children as left son of the clone. Last child is placed as
right son of last clone.

Let's begin designing our algorithm by writing recursive formula for
minimal cost inclined by placing virtual machines in leaves of a given
tree. Our approach is to evaluate this function using bottom-up
technique using auxilary array, which yields a dynamic programming
solution. To find actual placements of virtual machines in addition to
the cost, we traverse the array backwards, following the path of
minimas.

Keep in mind that number of virtual machines is equal to number of
chunks. However, our function $f$ will be defined by structural
induction on the tree and we will invalidate the property of having
the same number of chunks and virtual machines in a given subtree (which is true when
we look at whole tree).

Let's define $f$ in following way. First argument is a subtree (with
available informations like number of chunks in its leaves), and the
second argument is number of virtual machines that we decided to place
in the subtree (given as first parameter). To calculate optimum
placement of $x$ virtual machines in subtree $T$ ($f(T, x)$) we will
consider every possible split of number $x$ into two positive integer
values: $l$ and $r = x - l$. We will place $l$ virtual machines in
left subtree of $T$ and $r$ virtual machines in right subtree of
$T$. Having such information allow us to compute how much cost we
incline through edge $e_1$ (which connects left subtree of $T$ to root
of $T$) and edge $e_2$ (which connects right subtree of $T$ to root of
$T$). In a given recursive call we charge only those two edges, rest
of edges will be charged is subsequent calls.

Our cost function consists of two factors. First one, communication cost
between virtual machines is easy to compute. We know how many virtual
machines are in left subtree, how many are in right subtree and how
many are in whole tree outside of $T$. For each pair of virtual
machines, first of which is in left subtree and second of which is in
right subtree, we charge $b_2 \cdot (w(e_1) + w(e_2))$. For each pair
of virtual machines, first of which is in left subtree and second of
which is outside $T$, we charge $b_2 \cdot w(e_1)$. Right subtree case
is symmetrical. Second factor of our cost function is the cost of
transferring chunks to virtual machines. Let's call number of chunks
in left subtree as $c_l$ and number of chunks in right subtree as
$c_r$. To incline minimal cost we connect chunks in given subtree to
virtual machines in the same subtree. If we can no longer do that,
because $v_i < c_i (i \in \{l,r\})$, then we connect leftover chunks
to virtual machines in second subtree of $T$. If we can no longer do
that, we connect leftover chunks outside of $T$. This strategy is
optimal, because connecting any other way can be amended (TODO: need
better argument here), inclining lower cost. Connections inside either
left or right subtrees inclines cost $0$ to edges $e_1$ and
$e_2$. Connections between left and right subtree incline cost $b_1
\cdot (w(e_1) + w(e_2))$. Connections from either subtree to outside
of $T$ inclines either $b_1 \cdot w(e_1)$ or $b_2 \cdot
w(e_2)$. Finally, we can write down our formula for $f$:

$$ f(T, x) = min_{l \in \{0, \ldots, x\}} \{ f(T_l, l) + f(T_r, x - l)
+ TransferCost + ConnectionCost\} $$

where $TransferCost$ and $ConnectionCost$ are constants independent of
$l$, and are defined in paragraph above. One simplifying observation
is that to calculate $ConnectionCost$ we can just use the absolute
value of difference
between number of chunks and number of virtual machines in a given
subtree, without knowing which is bigger, because in our model if
there are some virtual machines left, we know that some chunks from
outside will use the same transfer as if we have excessive chunks in
the subtree.

Regarding base case we
trivially define leaf case as having cost $0$ if $x = 0$, cost $b_2
\cdot (n-1) + b_1\cdot n$ if there is no chunk in the leaf, cost $b_2 \cdot (n-1) +
b_1 \cdot (n-1)$ if there is a chunk in the leaf and $\infty$ otherwise.

Capacity constraints are preserved in such way that we put $f(T, x) =
\infty$ if either $e_1$ or $e_2$ transfer cost added to communication
cost exceedes its capacity. Doing so guarantees that this
(impossible) case can never be chosen as a minimum on higher levels of
recurrence calls (unless all other ways are impossible as well). 

When it comes to time complexity of described algorithm, we spend
certain amount of time in every of $2|T|$ vertices of binary
tree (2 is there because of binary transformation). This time can be
bound by iterating over splits of $n$ into two integers, times some
constant and we do it for every possible number of VMs from $0$ to $n$. Therefore, resulting running time is $O(Nn^2)$.
\end{comment}
