Model:
\begin{itemize}
  \item We are given a tree $T$ with capacities $c_e$ on the edges
    (physical network)easured in number of edges)
  \item We are given $n$ chunks that are assigned to leaves of $T$ (at
    most one chunk to one leaf)
  \item Our task is to place $n$ virtual machines on leaves of $T$ (at
    most one virtual machine to one leaf)
  \item Virtual machines can be placed directly on empty leaves or on leaves with chunks
  \item Each chunk has to be transferred to a virtual machine (matching), and the
    cost inlined by that is equal to $b_1$ times distance between
    chunk and virtual machine measured by number of edges
  \item Each virtual machine has to communicate with each other
    virtual machine, and the cost inclined by that is equal to $b_2$
    times distance between each pair of virtual machines meas total cost
\end{itemize}

Let's start by transforming our tree to binary tree with arbitrary
depth. We also introduce weights on edges (either $0$ or $1$). The
strategy we use is to clone every vertex $|children(v)| - 2$ times,
placing subsequent clones as right son of the previous one and placing
subsequent children as left son of the clone. Last child is placed as
right son of last clone.

Let's begin designing our algorithm by writing recursive formula for
minimal cost inclined by placing virtual machines in leaves of a given
tree. Our approach is to evaluate this function using bottom-up
technique using auxilary array, which yields a dynamic programming
solution. To find actual placements of virtual machines in addition to
the cost, we traverse the array backwards, following the path of
minimas.

Keep in mind that number of virtual machines is equal to number of
chunks. However, our function $f$ will be defined by structural
induction on the tree and we will invalidate the property of having
the same number of chunks and virtual machines in a given subtree (which is true when
we look at whole tree).

Let's define $f$ in following way. First argument is a subtree (with
available informations like number of chunks in its leaves), and the
second argument is number of virtual machines that we decided to place
in the subtree (given as first parameter). To calculate optimum
placement of $x$ virtual machines in subtree $T$ ($f(T, x)$) we will
consider every possible split of number $x$ into two positive integer
values: $l$ and $r = x - l$. We will place $l$ virtual machines in
left subtree of $T$ and $r$ virtual machines in right subtree of
$T$. Having such information allow us to compute how much cost we
incline through edge $e_1$ (which connects left subtree of $T$ to root
of $T$) and edge $e_2$ (which connects right subtree of $T$ to root of
$T$). In a given recursive call we charge only those two edges, rest
of edges will be charged is subsequent calls.

Our cost function consists of two factors. First one, communication cost
between virtual machines is easy to compute. We know how many virtual
machines are in left subtree, how many are in right subtree and how
many are in whole tree outside of $T$. For each pair of virtual
machines, first of which is in left subtree and second of which is in
right subtree, we charge $b_2 \cdot (w(e_1) + w(e_2))$. For each pair
of virtual machines, first of which is in left subtree and second of
which is outside $T$, we charge $b_2 \cdot w(e_1)$. Right subtree case
is symmetrical. Second factor of our cost function is the cost of
transferring chunks to virtual machines. Let's call number of chunks
in left subtree as $c_l$ and number of chunks in right subtree as
$c_r$. To incline minimal cost we connect chunks in given subtree to
virtual machines in the same subtree. If we can no longer do that,
because $v_i < c_i (i \in \{l,r\})$, then we connect leftover chunks
to virtual machines in second subtree of $T$. If we can no longer do
that, we connect leftover chunks outside of $T$. This strategy is
optimal, because connecting any other way can be amended (TODO: need
better argument here), inclining lower cost. Connections inside either
left or right subtrees inclines cost $0$ to edges $e_1$ and
$e_2$. Connections between left and right subtree incline cost $b_1
\cdot (w(e_1) + w(e_2))$. Connections from either subtree to outside
of $T$ inclines either $b_1 \cdot w(e_1)$ or $b_2 \cdot
w(e_2)$. Finally, we can write down our formula for $f$:

$$ f(T, x) = min_{l \in \{0, \ldots, x\}} \{ f(T_l, l) + f(T_r, x - l)
+ TransferCost + ConnectionCost\} $$

where $TransferCost$ and $ConnectionCost$ are constants independent of
$l$, and are defined in paragraph above. One simplifying observation
is that to calculate $ConnectionCost$ we can just use the absolute
value of difference
between number of chunks and number of virtual machines in a given
subtree, without knowing which is bigger, because in our model if
there are some virtual machines left, we know that some chunks from
outside will use the same transfer as if we have excessive chunks in
the subtree.

Regarding base case we
trivially define leaf case as having cost $0$ if $x = 0$, cost $b_2
\cdot (n-1) + b_1\cdot n$ if there is no chunk in the leaf, cost $b_2 \cdot (n-1) +
b_1 \cdot (n-1)$ if there is a chunk in the leaf and $\infty$ otherwise.

Capacity constraints are preserved in such way that we put $f(T, x) =
\infty$ if either $e_1$ or $e_2$ transfer cost added to communication
cost exceedes its capacity. Doing so guarantees that this
(impossible) case can never be chosen as a minimum on higher levels of
recurrence calls (unless all other ways are impossible as well). 

When it comes to time complexity of described algorithm, we spend
certain amount of time in every of $2|T|$ vertices of binary
tree (2 is there because of binary transformation). This time can be
bound by iterating over splits of $n$ into two integers, times some
constant and we do it for every possible number of VMs from $0$ to $n$. Therefore, resulting running time is $O(Nn^2)$.